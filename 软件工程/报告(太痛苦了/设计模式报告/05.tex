\subsection{服务端硬件配置需求}
\subsubsection{性能需求}
系统的设计必须保证在任何情况下都能正常运行,确保系统的稳定性,防止出现崩溃或其他故障。系统应能够快速响应用户的合法操作,对于常规操作,响应时间应在2秒内,即使在用户访问高峰时段,响应时间也不应超过5秒。此外,系统应能够及时检测并报告各种非正常情况,例如设备通信中断或无法连接数据库服务器等,以避免用户长时间等待。为了保证系统资源的有效利用,CPU的占用率应控制在合理范围内,内存占用率也应维持在设定的阈值以下。

\subsubsection{输入要求}
系统应具备有效的输入验证机制,以确保用户输入的数据符合预定的格式和长度要求。当用户输入事件标题和内容时,系统应自动检查其长度是否在允许的范围内。在基本模式下,用户必须设置事件类型才能成功创建事件;而在专家模式下,用户需要设置提醒时间才能创建事件。此外,系统应通过程序控制来降低出错几率,减少因用户操作失误导致的系统破坏。

通过满足这些性能需求和输入要求,我们的医疗预约系统将能够为病人提供一个全面、便捷的医疗服务体验。系统将支持病人进行注册登录、查看个人信息、向AI咨询病情、查看医院科室信息、选择并预约医生、提交和查看账单、缴纳费用、查询处方和搜索相关病历等功能。这些功能旨在提高医疗服务的可及性和效率,确保病人能够轻松管理自己的医疗需求。

\subsection{数据管理需求}
\subsubsection{系统独立性和完整性}
系统需要与其他系统进行有效的接口交互,同时保证本系统的独立性和完整性。必须防止未经授权的人员对系统进行设置修改或统计操作,确保系统的安全性和数据的准确性。

\subsubsection{数据备份与恢复}
服务器软件必须提供可靠的数据备份和恢复机制。在软硬件故障情况下,系统应能够利用备份数据和必要信息迅速恢复运行环境。用户信息管理模块的安全性对其他系统至关重要,必须确保统计数据的准确性和用户信息的完整备份及恢复。

\subsubsection{安全性保障}
系统应实施加密措施,包括登录数据加密和传输数据加密,以保障数据在系统间传输过程中的保密性和安全性。

\subsubsection{具体细则}
\begin{itemize}
	\item 服务器应具备至少20GB的存储空间。
	\item 数据库表的最大行数应支持达到1000行。
	\item 日志等记录的数据增长预计为每月10MB,具体速度取决于用户使用频率和业务数据量。
	\item 系统上线和学期初预计增长约500MB数据,具体量取决于业务数据量。
	\item 系统管理员应至少每两个月维护备份一次数据。
	\item 重大事故导致数据丢失时,系统应能在48小时内恢复数据。
	\item 系统崩溃后,应能在48小时内恢复运行。
\end{itemize}

\subsection{权限与安全需求}
安全性是系统正常运行的关键因素。因此,我们对系统中的安全与权限进行了以下设计:
\begin{itemize}
	\item 所有涉及功能信息或个人信息的网络事务都应进行加密操作。
	\item 用户不得非法修改数据库。
	\item 仅系统管理员有权查看系统日志。
	\item 任何人不得修改或删除日志。
	\item 仅系统管理员有权查看及修改底层数据库数据,且行为应被系统日志记录。
	\item 系统管理员应被允许进行数据的备份和恢复,以防止数据破坏和丢失。
	\item 系统应记录所有运行时发生的错误,包括本机错误和网络错误,以及用户的关键操作信息。
	\item 系统应具备数据加密传输和存储的安全保障。
	\item 系统基于开放的操作系统平台和数据库,因此需要建立相应的安全保障体系。
	\item 对可能造成严重后果的操作要有补救措施,提供取消功能,防止长时间等待。
	\item 对特殊符号和计算机代码的输入进行判断,阻止冲突字符的输入。
	\item 支持错误操作的可逆性处理,如取消系列操作,并在输入有效性字符前阻止后续操作。
\end{itemize}

\subsection{软件质量属性}
\subsubsection{鲁棒性}
系统必须具备强大的错误处理能力,能够妥善处理运行中可能出现的各种异常情况。这包括但不限于在创建事件、添加动作、检测地点时遇到的系统异常退出和连接断开问题。

\subsubsection{可用性}
系统设计要求能够持续稳定运行,确保7天24小时的服务可用性。全年累计的故障停运时间不得超过10小时,以提供无缝的医疗服务体验。

\subsection{可视化需求}
为了提升用户体验和界面友好性,系统将提供直观的操作结果反馈:
\begin{itemize}
	\item 用户添加事件后,应能在提醒列表中立即查看所创建事件的内容和提醒方式。
	\item 用户在选择相对时间计算时,应能清晰看到计算进度。
	\item 用户编辑事件、地点、响铃和振动设置后,应能立即看到更新后的结果。
\end{itemize}

\subsection{防护性需求}
系统需要具备一系列的防护措施,以防止操作失误和恶意攻击:
\begin{itemize}
	\item 在数据库发生误删除时,系统应提供撤销删除的功能以修复错误。
	\item 在重复操作导致系统卡死时,系统应发出警告。
	\item 当用户尝试访问无权限的资源时,系统应发出提示并禁止访问。
	\item 系统应定期进行信息备份,以防止病毒攻击。
	\item 系统应能够检测并防范恶意操作。
	\item 当检测到重复操作次数过多时,系统应发出警告,并在一段时间内禁止进一步操作。
\end{itemize}

\subsection{可维护性}
为了保证系统长期的正确性和稳定性,我们计划每周安排1到2个小时进行系统维护。维护的具体时间将根据用户一周内的访问次数统计数据来确定,以最小化维护工作对用户带来的不便。

通过满足这些要求,我们的医疗预约管理系统将能够为病人提供全面、便捷的医疗服务体验,从而显著提高医疗服务的可及性和效率。系统将支持病人进行注册登录、查看个人信息、向AI咨询病情、查看科室信息、预约挂号、提交和查看账单、缴纳费用、查询处方和搜索相关病历等功能,确保病人能够轻松管理自己的医疗服务需求。

\subsection{网络要求}
系统依赖于一个高度稳定的互联网连接,以及充足的带宽资源,以支持大量用户的同时在线访问。这包括:
\begin{itemize}
	\item 网络连接的冗余设计,以防单点故障。
	\item 网络流量的监控和优化,确保数据传输的高效性。
	\item 负载均衡机制,合理分配用户请求至多个服务器。
	\item 网络防火墙和入侵检测系统,增强网络安全性。
\end{itemize}

\subsection{服务端软件依赖要求}
\subsubsection{后端}
服务端后端的开发和部署需要以下软件依赖:
\begin{itemize}
	\item Java Development Kit (JDK) 8或以上版本,确保Java应用的兼容性和性能。
	\item Spring Boot 2.x框架,简化应用开发和部署。
	\item 应用服务器如Tomcat或Jetty,用于运行后端服务。
	\item 日志管理工具,如Log4j或SLF4J,用于记录和监控系统运行情况。
\end{itemize}

\subsubsection{数据库}
数据库的选择和管理需要考虑以下方面:
\begin{itemize}
	\item MySQL 5.7或以上版本,提供稳定的数据存储和查询性能。
	\item 数据库性能调优,包括索引优化和查询优化。
	\item 数据库备份和恢复策略,防止数据丢失。
	\item 数据安全措施,如加密和访问控制。
\end{itemize}

\subsubsection{其他依赖}
项目的构建和版本控制依赖于:
\begin{itemize}
	\item Maven或Gradle作为项目构建工具,自动化构建过程。
	\item Git作为版本控制工具,管理代码变更和协作开发。
	\item 依赖管理工具,如Gradle's Dependency Management或Maven's Dependency Injection,确保依赖库的一致性和更新。
\end{itemize}

\subsection{客户端软硬件配置要求}
\subsubsection{客户端}
客户端应兼容以下环境和配置:
\begin{itemize}
	\item 支持最新版本的主流Web浏览器,如Chrome、Firefox、Safari、Edge等,确保良好的用户体验。
	\item 对移动设备和平板电脑的优化,提供响应式设计。
	\item 客户端安全措施,如SSL/TLS加密,保护用户数据传输。
\end{itemize}

\subsubsection{硬件}
客户端硬件应满足以下最低配置要求:
\begin{itemize}
	\item CPU:至少双核1.5GHz以上,以处理计算密集型任务。
	\item 内存:至少2GB以上,支持多任务和快速响应。
	\item 存储空间:足够的硬盘空间用于缓存和日志文件。
	\item 网络适配器:支持有线和无线连接,确保网络稳定性。
\end{itemize}

\subsection{变更控制规范}
变更控制的详细规范包括:
\begin{itemize}
	\item 变更请求的提交、评审和批准流程。
	\item 变更实施前后的全面测试计划。
	\item 变更日志的详细记录,包括变更描述、影响范围和实施结果。
	\item 变更回滚策略,以应对可能的实施问题。
\end{itemize}

\subsection{故障处理}
故障处理的详细步骤包括:
\begin{itemize}
	\item 紧急故障的快速响应机制和流程。
	\item 故障影响评估和优先级分类。
	\item 故障处理团队的组成和职责分配。
	\item 故障解决方案的记录和分享。
\end{itemize}

\subsection{故障确认}
故障确认的详细流程包括:
\begin{itemize}
	\item 故障修复后的自动化测试脚本运行。
	\item 用户反馈收集和验证。
	\item 故障修复报告的编写和审批。
\end{itemize}

\subsection{故障自动恢复}
故障自动恢复机制的详细设计包括:
\begin{itemize}
	\item 自动化监控系统,实时检测故障迹象。
	\item 自动化恢复脚本和流程。
	\item 恢复过程的日志记录和审计。
\end{itemize}

\subsection{故障连锁诊断}
故障连锁诊断的详细步骤和机制包括:
\begin{enumerate}
	\item 异常检测的自动化工具和阈值设置。
	\item 故障定位的逻辑和诊断工具。
	\item 自动修复的策略和脚本。
	\item 报警通知的接收人和通知方式。
	\item 故障诊断报告的生成和存储。
\end{enumerate}